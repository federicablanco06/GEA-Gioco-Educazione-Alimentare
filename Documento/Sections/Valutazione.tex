\section{Valutazione} \label{sec:val}

%- Research question and variables
%- Participants profile
%- Context (where and when)
%- Data gathering
%- Execution
%- Results and discussion
La valutazione riguardante \acs{gea} è stata svolta fin dall'inizio per capire se l'idea alla base fosse ritenuta utile e se la struttura immaginata potesse soddisfare ogni bisogno.
\begin{enumerate}
\item Prima fase di valutazione:
		\begin{itemize}
		\item Domande: 	\begin{itemize}
						\item[*] Dove viene svolto il laboratorio di alimentazione?
						\item[*] In che modo viene svolto il laboratorio di alimentazione?
						\item[*] Che tipologie di argomenti vengono trattate?
						\item[*] Quali materiali vengono utilizzati?
						\item[*] Quale livello di difficoltà viene raggiunto?
						\item[*] Quali sono gli argomenti in cui si riscontrano maggiori difficoltà?
						\item[*] Qual è il rapporto dei ragazzi con le nuove tecnologie (es. Visore VR)?
						\item[*] Che cosa bisogna cercare di evitare o limitare?
						\item[*] \acs{gea} può risultare utile?
						\item[*] Come viene reputata la suddivisione in tre mini-giochi?		
						\end{itemize}
		\item Partecipanti: Terapiste e ragazzi, affetti da sindrome di \acs{ndd}, del centro "Fraternità 								e amicizia"
		\item Contesto: Il giorno 8/11/2017 in un'aula del centro "Fraternità e amicizia"
		\item Esecuzione: Sono state rivolte le domande rispettivamente a terapiste e ragazzi, i quali 								  hanno collaborato attivamente con molto entusiasmo, e in seguito è stata esposta 						  l'idea di \acs{gea} ed espresse le opinioni in merito 
		\item Risultato: L'idea è stata accolta con molto entusiasmo per cui si decide di proseguire  
		\end{itemize}
\newpage
\item Seconda fase di valutazione:
		\begin{itemize}
		\item Domande:  Chiesta un'opinione per quanto riguarda grafica, ambientazione, contenuti e 								strutturazione dei giochi
		\item Partecipanti: Eleonora, terapista del centro "Fraternità e amicizia"
		\item Contesto: Il giorno 23/11/2017 in un'aula del laboratorio I3Lab presso il Politecnico di 								Milano
		\item Esecuzione: Sono stati mostrati i mockup, riportati anche in questo documento, ad Eleonora e 						  rivolte le domande in merito
		\item Risultato: Ci sono stati forniti suggerimenti riguardanti la grafica e sottolineato il fatto 						 che non tutti i ragazzi sono in grado di leggere   
		\end{itemize}
\end{enumerate}


%A critical reflection on your work (challenges, critical aspects, main difficulties encountered, potential of the
%technology you have developed to address the needs of your target group and maybe of other targets…)
%• Why is this a «good» solution?
%• Are there competitors? If yes, why is your solution better than the existing ones?