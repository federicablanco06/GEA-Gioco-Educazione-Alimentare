\section{Stato dell'Arte} \label{sec:stato}
I grandi benefici dell'utilizzo della Realtà Virtuale in un contesto educativo e riabilitativo sono ormai riconosciuti a livello mondiale e provati tramite vari test comparativi tra riabilitazione con l'uso di nuove tecnologie e riabilitazione con l'uso di metodi classici \cite{rif3}, \cite{rif4}, \cite{rif5}. Negli ultimi anni si è registrato un incremento di interesse nell'utilizzo della \acs{vr} soprattutto nel campo degli \acs{ndd}, \cite{rif6} e \cite{rif7}, in quanto, come affermato in \cite{rif11}, sia la forza che le limitazioni della \acl{vr} sembrano adattarsi bene alle necessità che i learning tool per questa tipologia di disabilità richiedono. La dottoressa Dorothy Strickland, Department of Computer Science
North Carolina State University, nel suo trattato sullo studio di una applicazione in \acs{vr} per bambini autistici, \cite{rif12}, afferma che tra i grandi benefici che si possono riscontrare si hanno: controllo sugli stimoli di input, piccole modifiche per raggiungere una generalizzazione, situazioni di apprendimenti sicure, trattamento personalizzato e apprendimento con la minima interferenza umana. Il nostro progetto sfrutta tutte queste potenzialità grazie anche alla propensione dei bambini nel voler avere un'esperienza con nuove strumentazioni tecnologiche quali visori \acs{vr} perchè risultano affascinanti e attrattivi. Citando alcune applicazioni sviluppate nella stessa direzione e di successo tra gli end users troviamo "Wildcard", \cite{rif13}, e "Xoom", \cite{rif14}, i quali hanno portato a un netto miglioramento nello sviluppo di una propria autonomia e capacità di gestirsi nelle situazioni reali e ad un aumento nel mantenimento dell'attenzione.\\
Per quanto riguarda il campo dell'alimentazione risultano essere presenti molti giochi per pc e da svariati studi si è potuto apprendere che l'apprendimento risulta migliorato se supportato da un'esperienza interattiva, \cite{rif8}. Importanti aziende si impegnano ogni anno nella promozione e sviluppo di giochi interattivi educativi per bambini in questo campo sia per l'ambito scolastico che per quello domestico, ne è un esempio Nestlé con il progetto "Nutrikid", \cite{rif17}, elaborato con la consulenza scientifica della NFI, Nutrition Foundation of Italy. Come si legge dal loro sito web "La Nutrition Foundation of Italy, è stata costituita giuridicamente come associazione non-profit nel Dicembre 1976, con l'obiettivo di attivare interazioni e collaborazioni con gli organi governativi, le università e con l'industria per contribuire allo sviluppo della ricerca scientifica, allo scambio di informazioni nel campo dell'alimentazione ed alla promozione di ricerche interdisciplinari in questo settore.", \cite{rif18}, per questa motivazione abbiamo tratto da essa grandi spunti durante lo sviluppo di \acs{gea}. Altri progetti in questo campo vengono promossi di continuo dalla FEI, Food Education Italy, fondazione di partecipazione accreditata presso il MIUR, Ministero Istruzione Università Ricerca, che vive di contributi volontari e si occupa di aiutare le scuole e gli insegnanti a sviluppare il loro ruolo di educatori alimentari, \cite{rif19}.\\
Per quanto riguarda la tecnologia dei visori di \acl{vr} presenti ora sul mercato si ha una netta divisione tra due correnti: visori embedded, quale HTC Vive,\cite{rif15}, e visori modulari, quale Google Cardboard, \cite{rif9}. Durante la realizzazione e le fasi di test di \acs{gea} abbiamo deciso di utilizzare la seconda delle due scelte in quanto risulta essere la più economica sul mercato e il fattore finanziario è di grande importanza visto che il gioco è pensato per essere integrato in programmi terapeutici esistenti e ampiamente adottato. Questo visore è composto da due lenti biconvesse montate su una struttura in plastica o cartone disponibile in diversi colori e forme. Lo smartphone posto all'interno di questa struttura mostra i contenuti visivi, suddividendoli in immagini bidimensionali a due dimensioni identiche, e l'interazione è ottenuta attraverso lo sguardo focalizzato. L'utente può navigare nel mondo virtuale ruotando la sua testa che di conseguenza ruoterà la scena virtuale proiettata sul display.\\
Per la replicazione dello schermo su un pc o televisore la tecnologia adottata è Google Chromecast, \cite{rif10}, scelta effettuata anche in questo caso su base economia e sulla facilità di apprendimento d'uso. Il componente deve essere collegato alla presa di corrente e alla presa \acs{hdmi} del dispositivo su cui si vuole la replicazione, lo smartphone, su cui viene installata l'applicazione Google Home, e il Google Chromecast devono poi essere connessi alla stessa rete WiFi.