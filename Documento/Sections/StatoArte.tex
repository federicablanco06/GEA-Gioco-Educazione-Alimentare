\section{Stato dell'Arte} \label{sec:stato}
I grandi benefici dell'utilizzo della Realtà Virtuale in un contesto educativo e riabilitativo sono ormai riconosciuti a livello mondiale e provati tramite vari test comparativi tra riabilitazione con l'uso di nuove tecnologie e riabilitazione con l'uso di metodi classici \cite{rif3}, \cite{rif4}, \cite{rif5}. Negli ultimi anni si è registrato un incremento di interesse nell'utilizzo della \acs{vr} soprattutto nel campo degli \acs{ndd}, \cite{rif6}, \cite{rif7}. Il nostro progetto si focalizza proprio su questo aspetto e sulla propensione dei bambini nel voler avere un'esperienza con strumentazioni quali visori \acs{vr} perchè risultano affascinanti e attrattivi.\\
Per quanto riguarda il campo dell'alimentazione risultano essere presenti molti giochi per pc e da svariati studi si è potuto apprendere che l'apprendimento risulta migliorato se supportato da un'esperienza interattiva, \cite{rif8}.\\
Per quanto riguarda la tecnologia adottata abbiamo deciso di utilizzare, durante la realizzazione e le fasi di test, il visore Google Cardboard \cite{rif9}, la soluzione più economica sul mercato, poiché il gioco è pensato per essere integrato in programmi terapeutici esistenti e ampiamente adottato. Questo visualizzatore è composto da due lenti biconvesse montate su una struttura in plastica o cartone disponibile in diversi colori e forme. Lo smartphone posto all'interno di questa struttura mostra i contenuti visivi, suddividendoli in immagini bidimensionali a due dimensioni identiche, e l'interazione è ottenuta attraverso lo sguardo focalizzato. L'utente può navigare nel mondo virtuale ruotando la sua testa, che di conseguenza ruoterà la scena virtuale proiettata sul display.\\
Per la replicazione dello schermo su un pc o televisore la tecnologia adottata è Google Chromecast \cite{rif10} in quanto risulta essere la soluzione più accessibile sia economicamente sia per facilità di apprendimento. Il componente deve essere collegato alla presa di corrente e alla presa \acs{hdmi} del dispositivo su cui si vuole la replicazione, lo smartphone, su cui viene installata l'applicazione Google Home, e il Google Chromecast devono poi essere connessi alla stessa rete WiFi.