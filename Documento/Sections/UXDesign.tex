\section{UXDesign} \label{sec:design}

\subsection{Approccio generale} \label{subsec:app}

\subsection{Scenari} \label{scenari}
Di seguito vengono riportati tre scenari a rappresentazione dei giochi che si possono fare con \acs{gea}, sono tutti e tre di tipo testuale.
\subsubsection{Scenario 1}
Maria, terapista di un centro terapeutico per persone con disabilità, si trova nel suo studio pronta ad accogliere Emanuele, bimbo affetto da \acs{ndd}, per proseguire il loro percorso di educazione alimentare. Durante questa fase del laboratorio Maria decide di far uso di \acs{gea}, gioco di realtà virtuale per l'educazione alimentare, partendo da un livello basso di difficoltà. All'arrivo del bambino essa dunque avvia l'applicazione sopracitata e grazie al touchscreen seleziona il livello "basso" tra i tre proposti dal gioco; inserisce quindi lo smartphone all'interno del visore indossato poi dal ragazzo e lo invita a scegliere, puntando con lo sguardo, uno dei tre giochi disponibili  intitolati "Impara con la piramide", "É sano o no?" e "A tavola!". Emanuele punta con lo sguardo il primo e davanti ai suoi occhi compare la schermata di gioco in cui si trova la piramide alimentare incompleta e intorno le due possibili opzioni di completamento. Egli sempre con lo sguardo fa la sua scelta e il gioco gli mostra la mascotte sorridente ad indicare la correttezza della risposta.
\subsubsection{Scenario 2}
Alessia, bambina con disabilità, si sta recando con la mamma presso il centro terapeutico in cui è in cura piena di gioia perchè è Venerdì e quindi sa che farà laboratorio di alimentazione interattivo usando un gioco chiamato \acs{gea}. Una volta arrivata indossa, come ormai ben sa, il visore passatole dalla sua terapista e con lo sguardo punta subito il suo gioco preferito: "É sano o no?". La schermata che le appare mostra nella parte sinistra due pietanze differenti e sulla parte destra una pattumiera, lo scopo del gioco è quello di "buttare", trascinandolo con lo sguardo, nella pattumiera il piatto ritenuto cibo "spazzatura". Alessia con lo sguardo butta purtroppo il cibo errato e le appare la mascotte con il volto triste ad indicare la scelta erronea; il gioco torna quindi sulla schermata dandole la possibilità di ritentare.
\subsubsection{Scenario 3}
Il papà di Alberto si reca insieme al figlio disabile presso il centro terapeutico perchè il bambino ha dei seri problemi di alimentazione ossia non riesce ad imparare quale pietanza sia adatta al pasto in considerazione. Sono ormai molte sedute che svolge con la sua terapista Dalila ed è arrivato il momento di rendere questo percorso di cura più interattivo grazie all'uso di \acs{gea}, un gioco per l'educazione alimentare. Dalila fa indossare ad Alberto il visore con l'applicazione avviata e invita il bambino a puntare con lo sguardo l'icona con la scritta "A tavola!". Davanti ai suoi occhi appare l'immagine di un bel piatto di pasta fumante con sotto le due opzioni: colazione e cena. Il bambino preso da entusiasmo esclama ad alta voce cena e con lo sguardo punta la casella corrispondente: appare così la mascotte del gioco con il volto sorridente a conferma della scelta effettuata.