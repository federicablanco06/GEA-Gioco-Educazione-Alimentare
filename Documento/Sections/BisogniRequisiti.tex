\section{Bisogni e Requisiti} \label{sec:biso}

\subsection{Target Group} \label{subsec:target}
\acs{gea} è rivolto ai bambini affetti da \acs{ndd} ossia da malattie neurologiche, in particolare i nostri principali utenti sono seguiti da terapisti che li aiutano nel loro percorso e portano avanti un progetto di educazione alimentare che si svolge durante un laboratorio con personale qualificato.

\subsection{Stakeholder e Bisogni} \label{subsec:stakbis}
I nostri stakeholder sono divisi essenzialmente in tre categorie:
\begin{itemize}
\item Utenti primari (end user): bambini affetti da sindrome \acs{ndd}
\item Utenti secondari: terapisti
\item Utenti terziari: familiari, sviluppatori, manager
\end{itemize}
Il bisogno che la nostra applicazione cerca di soddisfare è un necessario sviluppo di un'autonomia personale e domestica nel campo dell'alimentazione: aumentare dunque la capacità di stabilire in autonomia quali cibi è corretto mangiare, in quante dosi e a quali pasti del giorno.


\subsection{Obiettivo del progetto} \label{subsec:obiettivo}
L'obiettivo del progetto \acs{gea} è creare un'applicazione in grado di insegnare ai bambini affetti da \acs{ndd} l'educazione alimentare tramite un'esperienza di gioco interattiva.


\subsection{Contesto} \label{subsec:contesto}
Il contesto di utilizzo di \acs{gea} è una stanza in un centro specializzato durante una seduta di laboratorio di alimentazione con la presenza di una terapista.


\subsection{Vincoli} \label{subsec:vincoli}
Vi sono svariati vincoli da dover rispettare:
\begin{itemize}
\item Basso costo
\item Grande facilità di utilizzo
\item Basso consumo di energia
\item Connessione internet attiva
\item Limitazioni sulla grafica: evitare colori freddi o alcuni tipi di animazioni
\item Necessità di spiegazioni visive in quanto alcuni utenti non sanno leggere
\end{itemize}


\subsection{Requisiti iniziali} \label{subsec:requis}
Per quanto riguarda i dispositivi necessitiamo di:
\begin{enumerate}
\item Smartphone con applicazione \acs{gea} installata
\item Visore di realtà virtuale
\item Pc
\item Google Chromecast
\end{enumerate}

Per quanto riguarda l'interazione con l'applicazione essa avverrà con due modalità differenti: ci sarà un'interazione di tipo touchscreen all'avvio di \acs{gea} per permettere alla terapeuta di scegliere il gioco da far svolgere al ragazzo e con quale difficoltà, l'interazione diventa poi di tipo visivo per la partita vera e propria in quanto le varie risposte verranno date con il movimento oculare.


\acs{gea} è un gioco con diverse funzionalità quali l'esercizio della memoria, abilità di problem solving, capacità di decision making e learning tool.

