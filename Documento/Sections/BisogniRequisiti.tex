\section{Bisogni e Requisiti} \label{sec:biso}

\subsection{Target Group} \label{subsec:target}
\acs{gea} è rivolto ai bambini affetti da \acs{ndd} ossia da malattie neurologiche, in particolare i nostri principali utenti sono seguiti da terapisti che li aiutano nel loro percorso e portano avanti un progetto di educazione alimentare che si svolge durante un laboratorio con personale qualificato.

\subsection{Stakeholder e Bisogni} \label{subsec:stakbis}
I nostri stakeholder sono divisi essenzialmente in tre categorie:
\begin{itemize}
\item Utenti primari (end user): bambini affetti da sindrome \acs{ndd}
\item Utenti secondari: terapisti
\item Utenti terziari: familiari, sviluppatori, manager
\end{itemize}
Il bisogno che la nostra applicazione cerca di soddisfare è un necessario sviluppo di un'autonomia personale e domestica nel campo dell'alimentazione: aumentare dunque la capacità di stabilire in autonomia quali cibi è corretto mangiare, in quante dosi e a quali pasti del giorno.


\subsection{Obiettivo del progetto} \label{subsec:obiettivo}
L'obiettivo del progetto \acs{gea} è creare un'applicazione in grado di insegnare ai bambini affetti da \acs{ndd} l'educazione alimentare tramite un'esperienza di gioco interattiva.


\subsection{Contesto} \label{subsec:contesto}
Il contesto di utilizzo di \acs{gea} è una stanza in un centro specializzato durante una seduta di laboratorio di alimentazione con la presenza di una terapista.


\subsection{Vincoli} \label{subsec:vincoli}
Vi sono svariati vincoli da dover rispettare:
\begin{itemize}
\item Basso costo
\item Grande facilità di utilizzo
\item Basso consumo di energia
\item Connessione internet attiva
\item Limitazioni sulla grafica: evitare colori freddi o alcuni tipi di animazioni
\item Necessità di spiegazioni visive in quanto alcuni utenti non sanno leggere
\end{itemize}

\subsection{Interazione e funzionalità} \label{subsec:intfun}
Per quanto riguarda l'interazione con l'applicazione essa avverrà con due modalità differenti: ci sarà un'interazione di tipo touchscreen all'avvio di \acs{gea} per permettere alla terapeuta di scegliere il gioco da far svolgere al ragazzo e con quale difficoltà, l'interazione diventa poi di tipo visivo per la partita vera e propria in quanto le varie risposte verranno date con il movimento oculare.\\

\acs{gea} è un gioco con diverse funzionalità quali l'esercizio della memoria, abilità di problem solving, capacità di decision making e learning tool.


\subsection{Requisiti} \label{subsec:requis}
I requisiti per \acs{gea} sono stati raccolti attraverso incontri con psicologi ed esperti nel campo della \acs{ndd}. Abbiamo anche preso parte ad alcune attività di laboratorio alimentare organizzate nel centro terapeutico che collabora con noi. Abbiamo osservato come vengono svolte le attività e abbiamo discusso con i pazienti (gli individui \acs{ndd} ad alto funzionamento) e i terapeuti del centro sull'idea di realizzare un gioco \acs{vr} con gli stessi contenuti e obiettivi. I primi incontri per la raccolta dei requisiti si sono svolti nel seguente modo:
\begin{enumerate}
\item Primo incontro:
		\begin{itemize}
		\item Domande: 	\begin{itemize}
						\item[*] Dove viene svolto il laboratorio di alimentazione?
						\item[*] In che modo viene svolto il laboratorio di alimentazione?
						\item[*] Che tipologie di argomenti vengono trattate?
						\item[*] Quali materiali vengono utilizzati?
						\item[*] Quale livello di difficoltà viene raggiunto?
						\item[*] Quali sono gli argomenti in cui si riscontrano maggiori difficoltà?
						\item[*] Qual è il rapporto dei ragazzi con le nuove tecnologie (es. Visore VR)?
						\item[*] Che cosa bisogna cercare di evitare o limitare?
						\item[*] \acs{gea} può risultare utile?
						\item[*] Come viene reputata la suddivisione in tre mini-giochi?		
						\end{itemize}
		\item Partecipanti: Terapiste e ragazzi, affetti da sindrome di \acs{ndd}, del centro "Fraternità 								e amicizia"
		\item Contesto: Il giorno 8/11/2017 in un'aula del centro "Fraternità e amicizia"
		\item Esecuzione: Sono state rivolte le domande rispettivamente a terapiste e ragazzi, i quali 								  hanno collaborato attivamente con molto entusiasmo, e in seguito è stata esposta 						  l'idea di \acs{gea} ed espresse le opinioni in merito 
		\item Risultato: L'idea è stata accolta con molto entusiasmo per cui si decide di proseguire  
		\end{itemize}
\item Secondo incontro:
		\begin{itemize}
		\item Domande:  Chiesta un'opinione per quanto riguarda grafica, ambientazione, contenuti e 								strutturazione dei giochi
		\item Partecipanti: Eleonora, terapista del centro "Fraternità e amicizia"
		\item Contesto: Il giorno 23/11/2017 in un'aula del laboratorio I3Lab presso il Politecnico di 								Milano
		\item Esecuzione: Sono stati mostrati i mockup, riportati anche in questo documento, ad Eleonora e 						  rivolte le domande in merito
		\item Risultato: Ci sono stati forniti suggerimenti riguardanti la grafica e sottolineato il fatto 						 che non tutti i ragazzi sono in grado di leggere   
		\end{itemize}
\end{enumerate}
 Dopo questi incontri, abbiamo quindi estratto i seguenti requisiti per l'applicazione da realizzare:
\begin{enumerate}
\item \textit{\textbf{Requisito 1: Personalizzazione}}\\
Il terapeuta deve essere in grado di impostare un livello di difficoltà nel gioco in base alle abilità del bambino in modo da adattare il gioco al suo livello di conoscenza e quindi aumentare gradualmente la difficoltà.
\item \textit{\textbf{Requisito 2: Contenuti pertinenti}}\\
Il contenuto del gioco deve essere intrinseco e riflettere quelli utilizzati durante il laboratorio alimentare. Per questo motivo, i giochi sviluppati dovrebbero basarsi sulla piramide alimentare, sul riconoscimento di cibi sani e sulla capacità di associare piatti e pasti della giornata.
\clearpage
\item \textit{\textbf{Requisito 3: Ambiente virtuale semplice}}\\
A causa delle varie disabilità che interessano gli utenti, è necessario tenere conto in particolare della grafica visualizzata: l'ambiente dovrebbe contenere solo elementi essenziali per l'attività specifica e colori freddi che dovrebbero essere evitati, così come animazioni improvvise o flash, in quanto potrebbero innescare reazioni negative.
\item \textit{\textbf{Requisito 4: Spiegazioni visive}}\\
I possibili utenti del gioco differiscono per età e gravità della disabilità, quindi c'è la possibilità che alcuni non siano in grado di leggere. Per questo motivo, ogni gioco deve includere spiegazioni visive sull'obiettivo e su come portarlo a compimento.
\item \textit{\textbf{Requisito 5: Importanza del feedback}}\\
L'azione di ogni utente durante il gioco deve ricevere il feedback corretto. Dare un feedback positivo quando l'azione giusta è compiuta e un feedback negativo quando viene commesso un errore aiuta a mantenere l'intervallo di attenzione dei bambini e il loro coinvolgimento nel gioco.
\item \textit{\textbf{Requisito 6: Monitoraggio}}\\
Il terapeuta deve sempre tenere sotto controllo ciò che il bambino sta facendo durante l'esperienza di gioco, al fine di seguire i suoi miglioramenti e le sue difficoltà ed essere così in grado di fornire le spiegazioni necessarie.
\end{enumerate}


