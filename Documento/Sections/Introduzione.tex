\section{Introduzione} \label{sec:intro}
Il termine disturbo neurologico, \acs{ndd}, racchiude al suo interno tutte le condizioni che sono causate da una disfunzione di una parte del cervello o del sistema nervoso che mostrano alcuni sintomi nello sviluppo fisico e psicologico del bambino \cite{rif1}. Tra le malattie più comuni troviamo l'Autismo, \acs{asd}, il deficit dell'attenzione e iperattività, \acs{adhd}, e la sindrome di Down \cite{rif2}. I bambini che soffrono di queste sindromi hanno bisogno di un aiuto nello sviluppo di abilità cognitive quali attenzione e linguaggio, abilità sociali quali la capacità di relazionarsi con gli altri e abilità di autonomia personale e domestica. La Realtà Virtuale è al giorno d'oggi di facile accesso perchè necessita dell'uso di uno smartphone, posseduto dalla maggior parte della popolazione, e di un visore VR facilmente reperibile a basso costo; inoltre questa tecnologia è molto migliorata  per cui non si hanno più problemi di sensazione di nausea o difficoltà nella messa a fuoco. I grandi benefici dell'utilizzo della Realtà Virtuale in un contesto educativo e riabilitativo sono ormai riconosciuti a livello mondiale e provati tramite vari test comparativi tra riabilitazione con l'uso di nuove tecnologie e riabilitazione con l'uso di metodi classici \cite{rif3}, \cite{rif4}, \cite{rif5}. Il nostro progetto si focalizza proprio su questo aspetto e sulla propensione dei bambini nel voler avere un'esperienza con strumentazioni quali visori \acs{vr} perchè risultano affascinanti e attrattivi. \acs{gea}, \acl{gea}, si pone come obiettivo quello di aiutare grazie all'uso della \acs{vr} i bambini a sviluppare una autonomia propria e domestica nel campo dell'alimentazione, di grandissima importanza al giorno d'oggi come dimostrato durante il recente Expo tenutosi a Milano. Grazie a questo gioco i bambini alleneranno la memoria per ricordare i vari livelli della piramide alimentare, impareranno a riconoscere i cibi sani da quelli cosiddetti "spazzatura" e capiranno come abbinare un pasto della giornata ad una pietanza e viceversa. \acs{gea} viene sviluppata come app per smartphone ma con la necessaria replicazione dello schermo sul computer della terapista in modo che possa seguire tutto quello che il bambino compie durante l'esperienza di gioco. Nelle sezioni successive saranno spiegati in maggiore dettaglio lo stato dell'arte, i requisiti e bisogni dei nostri stakeholder, le tecnologie hardware e software utilizzate per lo sviluppo e gli scenari rappresentativi.