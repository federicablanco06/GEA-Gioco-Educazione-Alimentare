\documentclass[a4paper, 12pt]{article}

\usepackage[protrusion=true,expansion=true]{microtype} % Better typography
\usepackage{graphicx} 
\usepackage{subfigure}
\usepackage{wrapfig} %in-line image

\usepackage[dvipsnames]{xcolor}
\usepackage{listings}
\usepackage[italian]{babel}
\usepackage[T1]{fontenc} 
\usepackage[utf8]{inputenc}

\usepackage[hyphens]{url}
\usepackage{quoting}
%\usepackage[unicode,pdftex,plainpages=false,linktoc=all, hyperindex,citecolor=green,urlcolor=blue ]{hyperref}
\quotingsetup{font=small}

\usepackage[%printonlyused
]{acronym}

\makeatletter
\newcommand{\myparagraph}[1]{\paragraph{#1}\mbox{}\\}
\renewcommand\@biblabel[1]{\textbf{#1.}} % Change the square brackets for each bibliography item from '[1]' to '1.'
\renewcommand{\@listI}{\itemsep=0pt} % Reduce the space between items in the itemize and enumerate environments and the bibliography

% \renewcommand{\abstractname}{\large Introduction} %%Renaming the abstract

%-----title at right corner---------------------------------------
\renewcommand{\maketitle}{ 
\begin{figure}[h]
\centering
\includegraphics[width=7cm]{Images/LogoPolimi}\\[.5cm]
\end{figure}

\vspace{50pt}

\begin{flushright} 
\@date 

{\huge\@title} 
\vspace{140pt} 


\begin{tabular}{ l l c}
\large Blanco &  \large Federica & \large 875487\\
\large Pennati & \large Giulia & \large 882962\\
\end{tabular}

\vspace{40pt} 
\end{flushright}
\centering
\textsl{\large Advanced User Interface Project}\\
2017/2018
}


\title{\textbf{GEA}\\[3mm]
Gioco Educazione Alimentare}

\author{Blanco Federica\\Pennati Giulia} 

\date{01 Febbraio 2018}

%----------------------------------------------------------------------------------------


\begin{document}

\begin{titlepage}

\thispagestyle{empty}
\maketitle 

\end{titlepage}

\clearpage
\begin{center}
\textbf{Abstract}\\
\end{center}
This document describes the design of \acs{gea}, a virtual reality application for children with \acl{ndd} that has been designed in collaboration with a team of therapists and children themselves. \acs{gea} wants to teach nutrition education through a virtual reality game focused on recognizing the right food for the right situation and avoiding waste. It is based on real laboratory activities that children do in a local specialized center. We have reproduced these activities in a smart and comfortable space, in which children can play the three different games proposed. The application allows to choose the level of difficulty, in order to customize the game based on each child's skills. \acs{gea} allows also to replicate what the child is watching in the virtual environment on an external device, so that the therapist can see what the child is doing and provide help if there are some difficulties.
\\
\\
\\
\begin{center}
\textbf{Il Team}\\
\end{center}
\vspace{20px}
\begin{center}
\begin{minipage}[c]{.40\textwidth}
\centering
\includegraphics[width=.70\textwidth]{Images/Fede.jpg}\\
\vspace{10px}
\emph{Federica Blanco\\ federica.blanco@mail.polimi.it\\ 3891098408}\bigskip
\end{minipage}
\hspace{10mm}
\begin{minipage}[c]{.40\textwidth}
\centering
\includegraphics[width=.70\textwidth]{Images/Giuli.jpg}\\
\vspace{10px}
\emph{Giulia Pennati\\ giulia1.pennati@mail.polimi.it\\ 3388074110}\bigskip
\end{minipage}
\end{center}

\clearpage


%\input{Acronym.tex}
%\renewcommand{\abstractname}{Summary} % Uncomment to change the name of the abstract to something else
%\vspace{30pt} % Some vertical space between the abstract and first section
\thispagestyle{empty}
\tableofcontents
\clearpage 
\thispagestyle{empty}
\listoffigures
\clearpage 
 

\newpage
\pagenumbering{arabic}
\section{Introduzione} \label{sec:intro}

(a summary of the key points of your project) 1-2 pag

Il termine disturbo neurologico, \acs{ndd}, racchiude al suo interno tutte le condizioni che sono causate da una disfunzione di una parte del cervello o del sistema nervoso che mostrano alcuni sintomi nello sviluppo fisico e psicologico del bambino. Tra le malattie più comuni troviamo l'Autismo, \acs{asd}, il deficit dell'attenzione e iperattività, \acs{adhd}, e la sindrome di Down. I bambini che soffrono di queste sindromi hanno bisogno di un aiuto nello sviluppo di abilità cognitive quali attenzione e linguaggio, abilità sociali quali la capacità di relazionarsi con gli altri e abilità di autonomia personale e domestica. I grandi benefici dell'utilizzo della Realtà Virtuale in un contesto educativo e riabilitativo sono ormai riconosciuti a livello mondiale. Il nostro progetto si focalizza proprio su questo aspetto. \acs{gea}, \acl{gea}, si pone come obiettivo quello di aiutare grazie all'uso della \acs{vr} i bambini a sviluppare una autonomia propria e domestica nel campo dell'alimentazione. Grazie a questo gioco i bambini alleneranno la memoria per ricordare i vari livelli della piramide alimentare, impareranno a riconoscere i cibi sani da quelli cosiddetti "spazzatura" e capiranno come abbinare un pasto della giornata ad una pietanza e viceversa. \acs{gea} viene sviluppata come app per smartphone ma con la necessaria replicazione dello schermo sul computer della terapista in modo che possa seguire tutto quello che il bambino compie durante l'esperienza di gioco. Nelle sezioni successive saranno spiegati in maggiore dettaglio lo stato dell'arte, i requisiti e bisogni dei nostri stakeholder, le tecnologie hardware e software utilizzate per lo sviluppo e gli scenari rappresentativi.
\newpage
\section{Stato dell'Arte} \label{sec:stato}
(related works in the research or market arena that address similar problems)
\newpage
\section{Bisogni e Requisiti} \label{sec:biso}

\subsection{Target Group} \label{subsec:target}


\subsection{Stakeholder e Bisogni} \label{subsec:stakbis}


\subsection{Obiettivo del progetto} \label{subsec:obiettivo}
L'obiettivo del progetto \acs{gea} è creare un'applicazione in grado di insegnare ai bambini affetti da \acs{ndd} l'educazione alimentare tramite un'esperienza di gioco interattiva.


\subsection{Contesto} \label{subsec:contesto}
Il contesto di utilizzo di \acs{gea} è una stanza in un centro specializzato durante una seduta di laboratorio di alimentazione con la presenza di una terapista.


\subsection{Vincoli} \label{subsec:vincoli}
Vi sono svariati vincoli da dover rispettare:
\begin{itemize}
\item Basso costo
\item Grande facilità di utilizzo
\item Basso consumo di energia
\item Connessione internet attiva
\item Limitazioni sulla grafica: evitare colori freddi o alcuni tipi di animazioni
\item Necessità di spiegazioni visive in quanto alcuni utenti non sanno leggere
\end{itemize}


\subsection{Requisiti} \label{subsec:requis}
\newpage
\section{UXDesign} \label{sec:design}

\subsection{Approccio generale} \label{subsec:app}
In questa sezione mostriamo i diagrammi di flusso rappresentativi dell'intera applicazione e dei singoli giochi.

La Figura \ref{fig:Diagramma di flusso generale} mostra il diagramma di flusso dell'intera applicazione: all'avvio dell'applicazione si hanno in sequenza la scelta del gioco, la scelta del livello di difficoltà e poi l'avvio della partita composta da una sessione di gioco per "Impara con la piramide" mentre da tre sessioni per gli altri due giochi. Al termine delle sessioni la partita viene chiusa mostrando il punteggio totalizzato.

La Figura \ref{fig:Diagramma di flusso del gioco "Impara con la piramide"} mostra il diagramma di flusso del gioco "Impara con la piramide": in sequenza si hanno la visualizzazione dell'intera piramide incompleta e di un puntatore sullo specifico livello da completare con l'apparizione delle tre opzioni tra cui scegliere. Nel caso di scelta corretta verrà mostrato il feedback corrispondente e incrementato il punteggio, nel caso di scelta erronea verrà solamente mostrato il feedback.

La Figura \ref{fig:Diagramma di flusso dei giochi "E' sano o no?" e "A tavola!"} mostra il diagramma di flusso dei giochi "E' sano o no?" e "A tavola!". Per entrambi verrà mostrata la schermata iniziale con le opzioni tra cui scegliere: nel caso di scelta corretta verrà mostrato il feedback corrispondente e incrementato il punteggio, nel caso di scelta erronea verrà solamente mostrato il feedback.

\begin{figure}[htbp]
\centering
\includegraphics[scale=0.6]{Images/Flussogenerico}
\caption{Diagramma di flusso generale}
\label{fig:Diagramma di flusso generale}
\end{figure}
\clearpage

\begin{figure}[htbp]
\centering
\includegraphics[scale=0.6]{Images/Flussopiramide}
\caption{Diagramma di flusso del gioco "Impara con la piramide"}
\label{fig:Diagramma di flusso del gioco "Impara con la piramide"}
\end{figure}
\clearpage

\begin{figure}[htbp]
\centering
\includegraphics[scale=0.7]{Images/Flusso23}
\caption{Diagramma di flusso dei giochi "E' sano o no?" e "A tavola!"}
\label{fig:Diagramma di flusso dei giochi "E' sano o no?" e "A tavola!"}
\end{figure}
\clearpage

\subsection{Scenari} \label{scenari}
Di seguito vengono riportati tre scenari a rappresentazione dei giochi che si possono fare con \acs{gea}, sono tutti e tre di tipo testuale.
\subsubsection{Scenario 1}
Maria, terapista di un centro terapeutico per persone con disabilità, si trova nel suo studio pronta ad accogliere Emanuele, bimbo affetto da \acs{ndd}, per proseguire il loro percorso di educazione alimentare. Durante questa fase del laboratorio Maria decide di far uso di \acs{gea}, gioco di realtà virtuale per l'educazione alimentare, partendo da un livello basso di difficoltà. All'arrivo del bambino essa dunque avvia l'applicazione sopracitata e grazie al touchscreen seleziona il gioco "Impara con la piramide" perchè ha notato che Emanuele ha difficoltà nell'imparare quali alimenti si trovano in ogni specifico livello della piramide alimentare. Seleziona poi il livello "Facile" tra quelli possibili presentati e inserisce lo smartphone nel visore che il bambino va ad indossare. Il gioco mostra ad Emanuele prima l'intera piramide, una freccetta che punta uno specifico livello dando tre possibili scelte di completamento. Il bambino effettua con lo sguardo la sua scelta che risulta essere corretta per cui appare la mascotte GEA col viso sorridente a conferma.
\subsubsection{Scenario 2}
Alessia, bambina con disabilità, si sta recando con la mamma presso il centro terapeutico in cui è in cura piena di gioia perchè è Venerdì e quindi sa che farà laboratorio di alimentazione interattivo usando un gioco chiamato \acs{gea}. Una volta arrivata indossa, come ormai ben sa, il visore passatole dalla sua terapista che aveva precedentemente selezionato il gioco "E' sano o no?" perchè Alessia pasticcia un po' troppo nella sua alimentazione. La schermata che le appare mostra nella parte sinistra due pietanze differenti e sulla parte destra una pattumiera, lo scopo del gioco è quello di "buttare", trascinandolo con lo sguardo, nella pattumiera il piatto ritenuto cibo "spazzatura". Alessia con lo sguardo butta purtroppo il cibo errato e le appare la mascotte con il volto triste ad indicare la scelta erronea.
\clearpage
\subsubsection{Scenario 3}
Il papà di Alberto si reca insieme al figlio disabile presso il centro terapeutico perchè il bambino ha dei seri problemi di alimentazione ossia non riesce ad imparare quale pietanza sia adatta al pasto in considerazione. Sono ormai molte sedute che svolge con la sua terapista Dalila ed è arrivato il momento di rendere questo percorso di cura più interattivo grazie all'uso di \acs{gea}, un gioco per l'educazione alimentare. Dalila avvia l'applicazione, seleziona il gioco "A tavola!", seleziona la difficoltà e fa indossare il visore ad Alberto. Davanti agli occhi del bambino appare l'immagine di un bel piatto di pasta fumante con sotto le quattro opzioni: colazione, pranzo, merenda e cena. Il bambino preso da entusiasmo esclama ad alta voce cena e con lo sguardo punta la casella corrispondente: appare così la mascotte del gioco con il volto sorridente a conferma della scelta effettuata.
\clearpage

\subsection{Design iniziale: Mockup}

Il mockup in \textbf{Figura \ref{fig:Impara con la piramide}} mostra la schermata che comparirà dopo aver scelto di giocare a "Impara con la piramide" e dopo aver selezionato il livello di difficoltà desiderato. La schermata che appare spiega molto brevemente quello che il bambino dovrà fare giocando e quindi l'obiettivo da raggiungere.\\
Il mockup in \textbf{Figura \ref{fig:Piramide}} mostra la schermata successiva a quella esplicativa per quanto riguarda il gioco "Impara con la piramide". Qui viene mostrata al bambino l'intera piramide alimentare che il bambino dovrà via via completare durante il gioco.\\
Il mockup in \textbf{Figura \ref{fig:Piramide proseguimento}} viene presentata dopo aver mostrato l'intera piramide. Qui un puntatore indicherà un preciso livello della piramide alimentare che deve essere completato e si mostrano al bambino tre alimenti tra cui dover scegliere per effettuare il corretto completamento.\\
Il mockup in \textbf{Figura \ref{fig:E' sano o no?}} mostra la schermata che comparirà dopo aver scelto di giocare a "E' sano o no?" e dopo aver selezionato il livello di difficoltà desiderato. La schermata che appare spiega molto brevemente quello che il bambino dovrà fare giocando e quindi l'obiettivo da raggiungere.\\
Il mockup in \textbf{Figura \ref{fig:Schermata "E' sano o no?"}} mostra la schermata successiva a quella esplicativa per quanto riguarda il gioco "E' sano o no?". Qui vengono mostrati al bambino due pietanze tra le quali deve scegliere il cibo "spazzatura" e "buttarlo" con lo sguardo nella pattumiera presente sulla destra della schermata.\\
Il mockup in \textbf{Figura \ref{fig:Scelta "E' sano o no?"}} mostra la schermata relativa al compimento di una scelta: nel caso presentato la scelta è errata.\\
Il mockup in \textbf{Figura \ref{fig:A tavola!}} mostra la schermata che comparirà dopo aver scelto di giocare a "A tavola!" e dopo aver selezionato il livello di difficoltà desiderato. La schermata che appare spiega molto brevemente quello che il bambino dovrà fare giocando e quindi l'obiettivo da raggiungere.\\
Il mockup in \textbf{Figura \ref{fig:Schermata "A tavola!"}} mostra la schermata successiva a quella esplicativa per quanto riguarda il gioco "A tavola!". Qui vengono mostrati al bambino una pietanza e due possibili pasti del giorno: egli deve scegliere qual è il pasto più adatto per consumare quella pietanza. Il gioco si può presentare anche nella forma opposta ossia scegliere fra due piatti quale sia più adatto per il pasto indicato.\\
Il mockup in \textbf{Figura \ref{fig:Scelta "A tavola!"}} mostra la schermata relativa al compimento di una scelta: nel caso presentato la scelta è corretta.

\begin{figure*}
 \begin{minipage}[c]{\columnwidth}
   \centering
   \includegraphics[width=8cm]{Images/Mockup/gioco1}
   \caption{Impara con la piramide mockup}
   \label{fig:Impara con la piramide}
 \end{minipage}
 \ \hspace{8mm} \hspace{8mm} \\\
 
 \begin{minipage}[c]{\columnwidth}
  \centering
   \includegraphics[width=8cm]{Images/Mockup/piramide}
   \caption{Piramide mockup}
   \label{fig:Piramide}
 \end{minipage}
 \ \hspace{8mm} \hspace{8mm} \\\
 
 \begin{minipage}[c]{\columnwidth}
  \centering
   \includegraphics[width=8cm]{Images/Mockup/piramide2}
   \caption{Piramide proseguimento mockup}
   \label{fig:Piramide proseguimento}
 \end{minipage}
\end{figure*}
\clearpage

\begin{figure*}
 \begin{minipage}[c]{\columnwidth}
   \centering
   \includegraphics[width=8cm]{Images/Mockup/gioco2}
   \caption{E' sano o no? mockup}
   \label{fig:E' sano o no?}
 \end{minipage}
 \ \hspace{8mm} \hspace{8mm} \\\
 
 \begin{minipage}[c]{\columnwidth}
  \centering
   \includegraphics[width=8cm]{Images/Mockup/sano}
   \caption{Schermata "E' sano o no?" mockup}
   \label{fig:Schermata "E' sano o no?"}
 \end{minipage}
 \ \hspace{8mm} \hspace{8mm} \\\
 
 \begin{minipage}[c]{\columnwidth}
  \centering
   \includegraphics[width=8cm]{Images/Mockup/sanoscelta}
   \caption{Scelta cibo "spazzatura" mockup}
   \label{fig:Scelta "E' sano o no?"}
 \end{minipage}
\end{figure*}
\clearpage

\begin{figure*}
 \begin{minipage}[c]{\columnwidth}
   \centering
   \includegraphics[width=8cm]{Images/Mockup/gioco3}
   \caption{A tavola! mockup}
   \label{fig:A tavola!}
 \end{minipage}
 \ \hspace{8mm} \hspace{8mm} \\\
 
 \begin{minipage}[c]{\columnwidth}
  \centering
   \includegraphics[width=8cm]{Images/Mockup/atavola}
   \caption{Schermata "A tavola!" mockup}
   \label{fig:Schermata "A tavola!"}
 \end{minipage}
 \ \hspace{8mm} \hspace{8mm} \\\
 
 \begin{minipage}[c]{\columnwidth}
  \centering
   \includegraphics[width=8cm]{Images/Mockup/atavolascelta}
   \caption{Scelta pasto mockup}
   \label{fig:Scelta "A tavola!"}
 \end{minipage}
\end{figure*}
\clearpage


\subsection{Design attuale}
Di seguito vengono riportati alcuni screenshot riguardati il design attuale di \acs{gea}.\\
Le due immagini qui riportate si riferiscono alla schermata di home di \acs{gea} in landascape e in portscape. Qui si può effettuare la scelta riguardante la tipologia di gioco che si vuole effettuare.

\begin{center}
\begin{minipage}[c]{.40\textwidth}
\centering
\includegraphics[width=.70\textwidth]{Images/Design/Game}\\
\vspace{10px}
\emph{Schermata iniziale pc}\bigskip
\end{minipage}
\hspace{10mm}
\begin{minipage}[c]{.40\textwidth}
\centering
\includegraphics[width=.70\textwidth]{Images/Design/Homesmartphone}\\
\vspace{10px}
\emph{Schermata iniziale smartphone}\bigskip
\end{minipage}
\end{center}

La prossima immagine si riferisce invece alla schermata successiva alla scelta di gioco che riguarda la possibilità di selezionare il livello di difficoltà desiderato oppure tornare alla schermata precedente.
\begin{figure}[htbp]
\centering
\includegraphics[width=.70\textwidth]{Images/Design/Level}\\
\vspace{10px}
\emph{Scelta livello di difficoltà}\bigskip
\end{figure}

I tre screenshot di seguito si riferiscono rispettivamente all'avvio dei tre giochi.
\begin{center}
\includegraphics[width=.70\textwidth]{Images/Design/pyramid}\\
\vspace{5px}
\emph{Gioco 1: Impara con la piramide!}\bigskip
\end{center}
\begin{center}
\includegraphics[width=.70\textwidth]{Images/Design/Healthy}\\
\vspace{5px}
\emph{Gioco 2: E' sano o no?}\bigskip
\end{center}
\begin{center}
\includegraphics[width=.70\textwidth]{Images/Design/Eat}\\
\vspace{5px}
\emph{Gioco 3: A tavola!}\bigskip
\end{center}
\clearpage

Le due immagini successive rappresentano i feedback mostrati dopo che una scelta viene effettuata: la mascotte del gioco apparirà con volto felice in caso di risposta corretta e con volto triste nel caso di risposta errata. Per quanto riguarda il primo gioco si avrà inoltre il rispettivo livello della piramide che si colora di verde o rosso a ricordare durante il resto della partita le scelte giuste o sbagliate.
\begin{center}
\includegraphics[width=.70\textwidth]{Images/Design/Correct}\\
\vspace{5px}
\emph{Feedback risposta corretta}\bigskip
\end{center}
\begin{center}
\includegraphics[width=.70\textwidth]{Images/Design/Wrong}\\
\vspace{5px}
\emph{Feedback risposta errata}\bigskip
\end{center} 
\newpage
\section{Implementazione} \label{sec:tecn}

\subsection{Architettura hardware} \label{subsec:hard}
I terapisti hanno bisogno di poter vedere quello che il bambino compie durante l'esperienza di gioco per cui abbiamo la necessità di replicare lo schermo dello smartphone su pc. Per questo la nostra architettura necessita l'utilizzo del dispositivo Google Chromecast che comunica con lo smartphone tramite Wi-Fi e viene inserito nella porta \acs{hdmi} del pc; smartphone e visore comunicano tramite \acs{usb}.
\vspace{70pt}
\begin{figure}[htbp]
\centering
\includegraphics[width=\textwidth]{Images/hardware}
\caption{Architettura hardware}
\label{fig:hardware}
\end{figure}
\clearpage

\subsection{Architettura software} \label{subsec:soft}
L'architettura software prevede la comunicazione di tre moduli principali: l'applicazione \acs{gea}, con cui il terapista (nella scelta iniziale del livello di difficoltà) comunica, che è collegata al programma di gestione di orientamento visivo (necessario per captare i movimenti e il focus del bambino) del \acs{vr} e che trasferisce i dati al \acs{pc} del terapista che può quindi monitorare l'avanzamento del gioco e dare perciò eventuali suggerimenti al paziente.
\vspace{70pt}
\begin{figure}[htbp]
\centering
\includegraphics[width=\textwidth]{Images/software}
\caption{Architettura software}
\label{fig:software}
\end{figure}
\clearpage

\subsection{Linguaggi di programmazione e software utilizzati} \label{subsec:ling}
\newpage
\section{Valutazione} \label{sec:val}
Al termine dello sviluppo di \acs{gea} siamo riusciti ad ottenere un incontro presso il centro terapeutico in modo da poter testare il gioco sul campo e ricevere feedback e suggerimenti da parte di terapeuti e ragazzi.\\
Il giorno 19 Gennaio 2018 in una stanza del centro terapeutico "Fraternità e amicizia" abbiamo effettuato tre sessioni di test: la prima è stata svolta facendo provare il gioco agli specialisti del centro mentre le successive due coinvolgevano gruppi da 6/7 ragazzi ciascuno. Grazie all'utilizzo di Google Chromecast abbiamo potuto replicare lo schermo dello smartphone su un televisore così che anche chi non stava giocando poteva vedere come funzionavano i singoli mini-giochi. Tutti i ragazzi sono stati entusiasti e hanno voluto sperimentare tutti i giochi, nessuno di loro ha avuto sensazioni di nausea o fastidi causati da questo primo approccio alla realtà virtuale. \acs{gea} ha dunque suscitato grande interesse e ammirazione da parte di tutti gli utenti i quali hanno espressamente detto di volerlo continuare ad usare nelle loro sedute. I tempi di apprendimento del funzionamento e di gioco sono risultati similari in tutti coloro che hanno sperimentato per cui si può affermare che sia di facile utilizzo e comprensione ad eccezione del mini-gioco "E' sano o no?" in quanto risulta più difficile capire come trascinare l'elemento da buttare nella spazzatura. Per quanto riguarda i punteggi ottenuti anch'essi si sono rivelati nella norma e spesso molto elevati. Abbiamo inoltre notato che anche i ragazzi che non stavano sperimentando in prima persona venivano coinvolti grazie alla visione del gioco sullo schermo del televisore ed incitavano il compagno nelle scelte da prendere. \acs{gea}, pensato inizialmente come gioco individuale, è risultato invece molto efficace anche come gioco di gruppo e di squadra basato sull'aiuto reciproco e la collaborazione. \\
Di seguito vi sono delle foto scattate durante la mattinata sopracitata e gentilmente concesse dal centro "Fraternità e amicizia".
\begin{figure*}
 \begin{minipage}[c]{\columnwidth}
   \centering
   \includegraphics[width=8cm]{Images/terap}
   \caption{Prima sessione di test: Una terapeuta prova il gioco alla nostra presenza}
   \label{fig:test1}
 \end{minipage}
 \ \hspace{8mm} \hspace{8mm} \\\
 
 \begin{minipage}[c]{\columnwidth}
  \centering
   \includegraphics[width=8cm]{Images/raga}
   \caption{Seconda sessione di test: Una ragazza prova il gioco mentre gli altri osservano il televisore}
   \label{fig:test2}
 \end{minipage}
 \ \hspace{8mm} \hspace{8mm} \\\
 
 \begin{minipage}[c]{\columnwidth}
  \centering
   \includegraphics[width=8cm]{Images/strumenti}
   \caption{Primo piano dello schermo del televisore}
   \label{fig:tele}
 \end{minipage}
\end{figure*}
\clearpage

La nostra soluzione risulta essere una buona soluzione per svariate motivazioni quali:
\begin{itemize}
\item[->] La realtà virtuale permette di avere un ampio database con la presenza di tutti i possibili alimenti senza dover usare vero cibo che andrebbe quindi poi sprecato;
\item[->] \acs{gea} risulta essere una soluzione compatta in quanto si necessita solamente di uno smartphone, al giorno d'oggi posseduto dalla maggior parte della popolazione, e un visore \acs{vr}, acquistabile con una spesa minima, per cui facilmente trasportabile;
\item[->] \acs{gea}, per via della facile reperibilità della tecnologia utilizzata, permette al bambino di poter continuare la sua educazione in campo alimentare anche in completa autonomia a casa propria senza dover aspettare di andare al centro nella giornata dedicata al laboratorio di alimentazione;
\item[->] Grazie alla presenza di una grafica simpatica e colorata questa applicazione permette di avere un'esperienza divertente motivando i bambini nell'imparare giocando;
\item[->] Grazie all'uso di Google Chromecast è possibile avere due sedute in una: una seduta individuale svolta dal ragazzo che indossa il visore e gioca e una seduta di gruppo per i ragazzi seduti attorno che osservano lo schermo del televisore. Permette quindi training individuale e di gruppo nello stesso momento.
\end{itemize}
\newpage
\section{Sviluppi Futuri} \label{sec:svifut}

\newpage
\section{Appendice} \label{sec:app}

%----------------------------------------------------------------------------------------
%\newpage
%\pagestyle{empty}
%\bibliographystyle{unsrt}
%\bibliography{sample}
%----------------------------------------------------------------------------------------

\end{document}