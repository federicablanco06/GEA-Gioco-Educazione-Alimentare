\documentclass[a4paper, 12pt]{article}
\usepackage[protrusion=true,expansion=true]{microtype} % Better typography
\usepackage{graphicx} 
\usepackage{subfigure}
\usepackage{wrapfig} %in-line image
\usepackage[dvipsnames]{xcolor}
\usepackage{listings}
\usepackage[italian]{babel}
\usepackage[T1]{fontenc} 
\usepackage[utf8]{inputenc}
\usepackage[hyphens]{url}
\usepackage{quoting}
\quotingsetup{font=small}
\usepackage[
]{acronym}
\makeatletter
\newcommand{\myparagraph}[1]{\paragraph{#1}\mbox{}\\}
\renewcommand\@biblabel[1]{\textbf{#1.}}
\renewcommand{\@listI}{\itemsep=0pt} 

\begin{document}
\begin{center}
\fontsize{12mm}{4mm}\selectfont\textbf{GEA}\\\
\end{center}
\begin{center}
\fontsize{7mm}{3mm}\selectfont\textit{Gioco Educazione Alimentare}
\end{center}
GEA, a virtual reality application for children with NeuroDevelopmental Disorders (i.e. Autism Spectrum Disorder (ASD) Attention Deficit and Hyperactivity Disorder (ADHD) and Down syndrome), aims to help children to develop their own and domestic autonomy in the field of nutrition.\\

When the application starts, the home page is shown, with the three available games, \textit{"Learn with the pyramid!"}, \textit{"Healthy or not?"} and \textit{"Let’s eat!”}. After that, the therapist chooses the level of difficulty of the selected game, the difficulty does not lie in the way the game is played or in its objective but in the type of food shown.  The choice of the game and the difficulty is made via touchscreen on the smartphone, before inserting it in the VR viewer. When the game starts the patient is immersed in the virtual space, which is the same for all the games and reproduces the real setting of the therapy: there is a room with a table, a fridge, a window, a door, a sofa and a kitchen, in order to keep the space as simple as possible but also inherent with nutrition. In front of the user a short explanation, textual and graphical,  for each game and a play button appear. Moreover, a fantasy character serving as mascot of the game was created, with the goal of making the application more fun and serving as visual feed-back after each user’s action. The character is dressed with food-related clothes (strawberries, pumpkin, leaves, ...), to enhance the theme of the game.\\

\textbf{"Lear with the pyramid!"}: this game aims to teach how to complete the food pyramid, by selecting which food goes in each level. In the virtual environment, there is a pyramid divided into five levels, with a pointer indicating which level the user is currently completing and a table with three options. The user has to focus on the correct choice for a certain time interval to give the answer, avoiding possible unwanted answers while the user is looking around in the environment. \\
\clearpage
\textbf{“Healthy or not?"}: this game is proposed to train patients in recognizing if a dish is healthy or not. When the game starts, two dishes appear on the table of the virtual room, with a bin and the visual explanation of how the game works: the user must select the “junk food” with the eyes and move it, by keeping the gaze focused on it, until it is thrown in the bin.\\

\textbf{"Let’s eat!"}: this game aims to teach how to associate meals of the day to specific dishes. In this case, the virtual environment presents four images representing the four main meals (breakfast, lunch, afternoon snack and dinner) and a dish: the goal is to select the correct meal in which the dish can be eaten. \\

At the end of each session is shown to the user the total scored points and the application turns back to the home page. 


\end{document}